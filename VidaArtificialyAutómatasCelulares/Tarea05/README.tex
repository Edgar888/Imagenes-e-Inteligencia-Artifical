\documentclass[12pt]{article}
\usepackage[utf8]{inputenc}
\usepackage[spanish]{babel}
\usepackage{listings}
\usepackage{xcolor}
\usepackage{graphicx}

\lstset{
    language=Python,
    basicstyle=\ttfamily\small,
    keywordstyle=\color{blue}\bfseries,
    commentstyle=\color{green!60!black},
    stringstyle=\color{red},
    frame=single,
    breaklines=true,
    showstringspaces=false,
    numbers=left,
    numberstyle=\tiny,
    numbersep=5pt,
}

\title{Documentación del programa Red Booleana de Kauffman}
\author{Edgar Montiel Ledesma}
\date{\today}

\begin{document}

\maketitle

\section{Introducción}
El programa implementa una \textbf{Red Booleana de Kauffman}, que es un modelo matemático utilizado para simular sistemas complejos con interacciones no lineales entre nodos. Además, incluye una interfaz gráfica desarrollada en \textit{Tkinter}, que permite interactuar con el modelo y visualizar los resultados.

\section{Descripción del código}
El programa está dividido en tres partes principales: la definición de la red booleana, las funciones de simulación, y la interfaz gráfica.

\subsection{Definición de la red booleana}
La red booleana se implementa en una clase llamada \texttt{RedBooleanaKauffman}. Cada red tiene los siguientes componentes:

\begin{itemize}
    \item \textbf{Nodos:} Representan las variables del sistema, y cada uno puede estar en un estado binario (\texttt{0} o \texttt{1}).
    \item \textbf{Conexiones:} Cada nodo está conectado a otros \texttt{k} nodos, definidos aleatoriamente.
    \item \textbf{Funciones booleanas:} Cada nodo tiene una función booleana que determina su estado en función de los estados de los nodos a los que está conectado.
\end{itemize}

\noindent El siguiente fragmento de código define la clase:

\begin{lstlisting}
class RedBooleanaKauffman:
    def __init__(self, num_nodos, k):
        self.num_nodos = num_nodos
        self.k = k
        self.estados = [random.choice([0, 1]) for _ in range(num_nodos)]
        self.conexiones = self._generar_conexiones()
        self.funciones_booleanas = self._generar_funciones_booleanas()
\end{lstlisting}

\subsection{Funciones de la clase}
\begin{itemize}
    \item \texttt{\_generar\_conexiones:} Genera aleatoriamente las conexiones entre los nodos.
    \item \texttt{\_generar\_funciones\_booleanas:} Asigna una función booleana a cada nodo. Cada función es una tabla de verdad generada aleatoriamente.
    \item \texttt{\_evaluar\_funcion\_booleana:} Evalúa el próximo estado de un nodo utilizando su función booleana.
    \item \texttt{actualizar:} Actualiza los estados de todos los nodos de la red.
    \item \texttt{simular:} Simula la evolución de la red durante un número determinado de pasos.
\end{itemize}

\subsection{Interfaz gráfica}
La interfaz gráfica permite al usuario ingresar parámetros y visualizar los resultados de la simulación. Se implementa utilizando \texttt{Tkinter}.
\begin{figure}[h]
    \centering
    \includegraphics[width=0.6\textwidth]{K1.png}
    \caption{Imagen del programa.}
    \label{fig:original}
\end{figure}

\begin{lstlisting}
ventana = Tk()
ventana.title("Red Booleana de Kauffman")
ventana.geometry("800x600")
\end{lstlisting}

\noindent Las funciones principales de la interfaz son:
\begin{itemize}
    \item \texttt{inicializar\_red:} Crea una nueva red con los parámetros definidos por el usuario.
    \item \texttt{simular\_red:} Realiza la simulación de la red y muestra los resultados en un cuadro de texto.
\end{itemize}
        
\subsection{Visualización de resultados}
El historial de simulación se muestra en un cuadro de texto dentro de la interfaz. Cada línea representa el estado de la red en un paso de tiempo.



\section{Ejecución del programa}
Al ejecutar el programa, la ventana principal muestra los controles para configurar la red:
\begin{itemize}
    \item \textbf{Número de nodos:} Total de nodos en la red.
    \item \textbf{Conexiones por nodo (k):} Número de conexiones que cada nodo tiene con otros.
    \item \textbf{Pasos de simulación:} Número de iteraciones para simular.
\end{itemize}

\begin{figure}[h]
    \centering
    \begin{minipage}{0.45\textwidth}
        \centering
        \includegraphics[width=\textwidth]{K2.png}
        \caption{Resultado.}
        \label{fig:resultado}
    \end{minipage}
    \hfill
    \begin{minipage}{0.45\textwidth}
        \centering
        \includegraphics[width=\textwidth]{K3.png}
        \caption{Simulación.}
        \label{fig:simulacion}
    \end{minipage}
\end{figure}


\noindent Al inicializar la red y ejecutar la simulación, se generan los estados de la red en cada paso.

\section{Conclusión}
Este programa permite explorar el comportamiento de redes booleanas de Kauffman y observar cómo evolucionan sus estados con el tiempo. Además, la interfaz gráfica hace que sea fácil de usar y comprender los resultados.

\end{document}
