\documentclass{article}
\usepackage[utf8]{inputenc}

\title{README: Programa de Harry Potter}
\author{Edgar Montiel Ledesma}
\date{}

\begin{document}

\maketitle

\section{Descripción}
Este programa en Python está basado en la temática de Harry Potter y contiene dos funciones principales relacionadas con las casas de Hogwarts y los hechizos famosos de la saga. Además, el programa tiene la capacidad de reescribir su propio código fuente en un archivo llamado \texttt{replicating.py}.

\section{Funciones}
\subsection{casa\_hogwarts(hechizo)}
Esta función toma como argumento un hechizo y devuelve la casa de Hogwarts a la que está asociado. Las asociaciones predefinidas son:
\begin{itemize}
    \item \texttt{"Expecto Patronum"} $\rightarrow$ Gryffindor
    \item \texttt{"Avada Kedavra"} $\rightarrow$ Slytherin
    \item \texttt{"Lumos"} $\rightarrow$ Ravenclaw
    \item \texttt{"Wingardium Leviosa"} $\rightarrow$ Hufflepuff
\end{itemize}
Si el hechizo no está en la lista, devuelve \texttt{"Casa desconocida"}.

\subsection{hechizo\_favorito(casa)}
Esta función toma como argumento una casa de Hogwarts y devuelve el hechizo favorito de esa casa. Las asociaciones predefinidas son:
\begin{itemize}
    \item \texttt{"Gryffindor"} $\rightarrow$ Expecto Patronum
    \item \texttt{"Slytherin"} $\rightarrow$ Avada Kedavra
    \item \texttt{"Ravenclaw"} $\rightarrow$ Lumos
    \item \texttt{"Hufflepuff"} $\rightarrow$ Wingardium Leviosa
\end{itemize}
Si la casa no está en la lista, devuelve \texttt{"Hechizo desconocido"}.

\subsection{escribir\_codigo()}
Esta función es responsable de reescribir el código fuente completo del programa y guardarlo en un archivo llamado \texttt{replicating.py}. Utiliza una plantilla interna del propio código y lo escribe en el nuevo archivo.

\section{Ejecución}
Para ejecutar el programa, sigue estos pasos:
\begin{enumerate}
    \item Guarda el código en un archivo llamado \texttt{Quine.py}.
    \item En la terminal, ejecuta el siguiente comando:
    \begin{verbatim}
    python3 Quine.py
    \end{verbatim}
    \item El programa mostrará ejemplos del hechizo favorito de Gryffindor y la casa que usa el hechizo \texttt{Avada Kedavra}.
    \item Además, el programa generará un archivo llamado \texttt{replicating.py} que contendrá una copia del código fuente original.
\end{enumerate}

\section{Requisitos}
Para ejecutar el programa, necesitas tener instalado:
\begin{itemize}
    \item Python 3.x
\end{itemize}

\end{document}
